\documentclass[11pt]{article}

\renewcommand{\thesection}{\Roman{section}} 
\renewcommand{\thesubsection}{\thesection-\Alph{subsection}}
\usepackage{amsfonts,amsmath,amssymb}
\usepackage{hyperref}
\usepackage{sectsty}
\usepackage{multicol}
\usepackage[margin=1in]{geometry}
\usepackage{amsfonts, amsmath, amssymb}
\usepackage[none]{hyphenat}
\usepackage{fancyhdr}
\usepackage{parskip}
\usepackage{float}
\usepackage[nottoc, notlot, notlof]{tocbibind}
\usepackage{graphicx}
\usepackage{apacite}
\usepackage{xcolor}
\usepackage{enumitem}

\pagestyle{fancy}
\fancyhead{}
\fancyfoot{}
\fancyhead[L]{\slshape{Proyecto Final Teoría de Grafos 2021-1}}
\fancyfoot[C]{\thepage}
%\renewcommand{\headrulewidth}{0pt}
%\renewcommand{\baselinestretch}{1.5}
\allsectionsfont{\centering}


\begin{document}
    \begin{center}
        \huge{\textbf{Optimización del torneo colombiano de fútbol mediante el coloreo de grafos}}\\[10pt]
        \small{\textbf{Santiago Alvarez Barbosa y Nicolás Velandia Sanabria}}\\[10pt]
        \small{\textbf{Universidad del Rosario, Matemáticas Aplicadas y Ciencias de la Computación}}\\[10pt]
        \today\\
        \rule{\textwidth}{0.5pt}
            %\begin{abstract}
                %\textcolor{red}{\textbf{\MakeUppercase{revisar ortografía y poner ciertas parte en modo matematico }}}\\
                Repositorio Github: \url{https://github.com/SantiagoAlvarezb/Graph_Theory_Final_Project}
            %\end{abstract}
        \rule{\textwidth}{0.5pt}
    \end{center}

    \begin{multicols}{2}
    \setcounter{page}{1}

        \section{Introducción}
            La coloración de grafos etiqueta los de vértices o aristas de un grafo G con diferentes colores. 
            Se colorea de tal manera que ningún par de vértices o par de aristas adyacentes tengan el mismo color.\\
            Frecuentemente, la aplicación de coloración de grafos se ve en problemas de horarios colegiales o 
            universitarios, donde su finalidad es poder crear un horario óptimo donde se cumplen todas las restricciones consideradas. 
            Teniendo esto en cuenta, podemos aplicar las técnicas de coloración de grafos en otros ámbitos similares. 
            Un ejemplo de esto es dentro del ámbito deportivo, específicamente en torneos todos contra todos. La mayoría de las 
            ligas deportivas del mundo se basan en estos torneos, que consisten en \textit{n} equipos, donde se requiere que cada equipo se 
            enfrente a los demás \textit{m} veces dentro de un número específico de rondas. Por lo tanto, podemos aplicar ciertas 
            restricciones a ligas deportivas actuales, y con el uso de la coloración de grafos, obtener una solución óptima 
            del horario deportivo. 

        \section{Problema}
            Nos enfocaremos en el Fútbol Profesional Colombiano (FPC), el cual es un torneo todos contra todos sencillo 
            (cada equipo juega contra el otro una vez), en la temporada 2021-1. Construiremos el grafo de tal forma que podamos 
            usar algoritmos para encontrar el número cromático, que en nuestro caso es el número mínimo de rondas que el torneo necesita.
            Adicional al problema de coloreo general, consideraremos ciertas restricciones reales que se pueden incorporar al grafo.\\[10pt]
            Ejemplos de restricciones que podríamos implementar:

            \begin{itemize}
                \item Minimizar el número de \textit{cortes} en el horario, es decir, minimizar el número de veces que un equipo juega local o de visitante consecutivamente.
                \item Evitar que ciertos partidos se jueguen en la misma fecha debido a que equipos pueden compartir estadio.
                \item Equipo que empieza de local debe terminar la temporada como visitante y viceversa.
                \item Ciertos equipos deben jugar en ciertas rondas. Un ejemplo de porque se aplica esta restricción es para incrementar los ingresos de televisión.
            \end{itemize}
   
            \subsection{Objetivos}
                Nuestro objetivo general es encontrar un horario válido para la liga profesional colombiana en la temporada 2021-1. Con el uso de la coloración 
                de grafos y sus técnicas.\\[10pt]
                Objetivos específicos:
                \begin{itemize}
                    \item Encontrar una forma de representar las restricciones especificadas dentro de nuestro grafo. 
                    \item Encontrar el número cromático de nuestro grafos con el uso de algoritmos de coloración.
                    \item Encontrar un horario óptimo donde se cumplan las restricciones propuestas.
                \end{itemize}

        \section{Marco Teórico}

            \begin{itemize}
                \item \textbf{Grafo:} Un grafo $G$ es una terna que consiste en un conjunto de vértices $V(G)$, un conjunto 
                de aristas $E(G)$ y una relación que asocia a cada arista con un par de vértices (extremos) 
                no necesariamente distintos.
                \item \textbf{Grafo Simple:} $G = (V,E)$ es un grafo sin bucles ni aristas múltiples, donde E es un conjunto 
                de pares no ordenados de vértices.
                \item \textbf{Número cromático:} El número cromático $\chi(G)$ es el mínimo número de colores necesarios 
                para etiquetar los vértices de $G$ de tal manera que vértices adyacentes reciben colores 
                distintos.  
                \item \textbf{Librerías:} Se usó el lenguaje Python para el desarrollo del proyecto, espesificamente se usaron dos librerias:
                \begin{enumerate}
                    \item \textit{NetworkX:} Librería de Python para el estudio de grafos y análisis de redes.
                    \item \textit{Matplotlib:} Librería de Python para la generación de gráficas.
                \end{enumerate}        
            \end{itemize}
        
        \section{Modelamiento}
            Enfocados en el FPC en la temporada 2021-1, aplicaremos ciertas especificaciones para el desarrollo del proyecto. A continuación se presentan dichos aspectos:
            \begin{itemize}
                \item  Se usarán \textbf{9 equipos}, cada uno representado por un número:
                \begin{itemize}[label=$\diamond$]
                    \item Santa fe: \textit{0}
                    \item Millonarios: \textit{1}
                    \item Atlético Nacional: \textit{2}
                    \item Deportivo Independiente Medellín: \textit{3}
                    \item Deportivo Cali: \textit{4}
                    \item América de Cali: \textit{5}
                    \item Boyacá Chicó: \textit{6}
                    \item Patriotas: \textit{7}
                    \item Junior: \textit{8}
                \end{itemize}  
                \item  De estos 9 equipos, 4 comparten 2 estadios, lo que significa que en total hay \textbf{7 estadios}.
                \item  Tendremos \textbf{4 clásicos}:
                \begin{enumerate}
                    \item  Millonarios vs. Santa Fe.
                    \item  Atlético Nacional vs. Deportivo Independiente Medellín.
                    \item  América de Cali vs. Deportivo Cali.
                    \item  Boyacá Chicó vs. Patriotas.
                \end{enumerate}
            \end{itemize}

            Con esto en cuenta podemos implementar un grafo general $G$, el cual se usará para 
            encontrar el número mínimo de rondas para la realización del torneo.\\
            Para el grafo $G$, los vértices serán etiquetados con un par ordenado (\textit{a},\text{b}) que indican un 
            partido entre el equipo \textit{a} y el equipo \textit{b}, con \textit{a} siendo local; mientras que las aristas unen 
            vértices que tienen equipos en común. Al finalizar el procesamiento del grafo $G$, nos enfocaremos en la representación de las restricciones
            sobre el mismo.\\[10pt]
            Tendremos en cuenta restricciones fuertes, que obligatoriamente se debe cumplir, y restricciones 
            suaves, las cuales se cumplirán en lo posible.\\[10pt]
            Condiciones fuertes:
            \begin{itemize}
                \item Equipos que comparten estadios no pueden jugar en la misma jornada como locales.
                \item \textbf{NO} hay fechas de clásicos.
            \end{itemize}
            Condición suaves:
            \begin{itemize}  
                \item Minimizar \textit{cortes}. Tener en lo posible un torneo donde cada equipo alterna entre jugar de local y visitante.     
            \end{itemize}

        \section{Solución Propuesta}
            Agregaremos las restricciones fuertes una por una al grafo $G$ para tener una representación parcial del horario e ir coloreando cada nuevo grafo con el uso de un algoritmo de coloración propio de \textit{Networkx} 
            para hallar el numero cromático $\chi(G)$. Posteriormente, se implementará la restricción suave para obtener el horario final óptimo factible. 
        
        \section{Implementación de la Solución}
            \textbf{Primer paso:}\\ 
            Se creó el grafo $G$, que representa la liga de fúbol sin ninguna restricción. Como fue mencionado anteriormente, los vértices representan todos los posibles partidos
            mientras que las aristas unen aquellos vértices que tienen equipos en común.
            Adicionalmente, se ejecutó la función de coloreo al grafo para tener uno horario de la liga base.
            \begin{figure}[H]
                \begin{center}
                    \includegraphics[width=2.5 in]{visualizacion_grafo_9_equipos_coloreo_square.png}  
                \caption{Grafo $G$ no restringido coloreado}
                \end{center}
            \end{figure}
            \textbf{Segundo paso:}\\
            Se implementó la primera restricción fuerte.
            Inicialmente se identificaron los equipos que comparten estadios, en nuestro caso son los equipos: \textit{0, 1, 2, 3}. 
            Con esto en mente, se recorrió el grafo $G$ y se agregaron aristas entre vértices que contenian a uno de estos cuatro equipos como local,
            de tal modo que no se puedan jugar estos partidos en la misma jornada. Al finalizar el proceso, volvemos a colorear el grafo $G$.
            \begin{figure}[H]
                \begin{center}
                    \includegraphics[width=2.5 in]{visualizacion_grafo_9_equipos_coloreo_r1_square.png}  
                \caption{Grafo $G$ coloreado implementando la primera restricción fuerte}
                \end{center}
            \end{figure}

            \textbf{Tercer paso:}\\ 
            Después siguió implementar la segunda restricción fuerte sobre el grafo $G$ obtenido en el paso previo.
            El proceso fue similar al anterior ya que se identificaron los vértices que contienen los 
            partidos denominados clásicos y después se agregaron aristas entre ellos. En la siguiente figura se aprecia el nuevo grafo $G$ coloreado.
            \begin{figure}[H]
                \begin{center}
                    \includegraphics[width=2.5 in]{visualizacion_grafo_9_equipos_coloreo_r2_square.png}  
                \caption{Grafo $G$ coloreado implementando las dos restricciones fuertes}
                \end{center}
            \end{figure}
            \textbf{Cuarto paso:}
            Teniendo un horario de la liga donde las dos restricciones fuertes estan activas, el último paso es definir un horario en el cual se minimice los \textit{cortes}.
            Esto se llevó al cabo escogiendo una fecha base e identificando los equipos que jugaran de local dentro de esta fecha. Con esto en cuenta, comparamos 
            la base con las demas fechas con el propósito de encontrar una donde los equipos que jugaron de local en la base, jugaran de visitante en la otra.
            Si este no era el caso, se escogia la fecha en el cual la mayoría de equipos locales en la base jugaran de visitante. Esta metodología se usó hasta completar un horario.   
            Cabe mencionar que esta metodologia no nos asegura la minimización de cortes para los equipos de forma individual, sino que minimiza los cortes de forma general para el horario. 

        \section{Resultados Obtenidos}
            Habiendo adquirido las representacion de la Liga Profesional de Fútbol Colombiano con el uso de grafos conteniendo las restricciones fuertes y haciendo el análisis de la 
            restricción suave para obtener un horario factible y óptimo, se obtuvieron los siguientes resultados.
            \begin{itemize}
                \item Número de fechas: 11
                \item Partidos por cada equipo: 8
                \item Fechas de descanso por cada equipo:3
                \item Número maximo de fechas consecutivas jugando como local: 3
                \item Número maximo de fechas consecutivas jugando como visitante: 2
            \end{itemize}

            En la figura 4, se muestra el formato local-visitante del horario mientras que la información general con respecto a los \textit{cortes} se encuentra mejor resumida por equipo en la tabla 1.
            \begin{figure}[H]
                \begin{center}
                    \includegraphics[width=3.0 in]{cortes.png}  
                \caption{Visualización del formato Local-Visitante}
                \end{center}
            \end{figure}
           
            \begin{table}[H]
                \centering
                \begin{tabular}{|c||c|c|c|}
                    \hline
                    Equipos & \textit{Cortes} & Locales & Visitantes\\ \hline
                    0 & 1 & 0 & 1\\ \hline
                    1 & 0 & 0 & 0\\ \hline
                    2 & 0 & 0 & 0\\ \hline
                    3 & 1 & 1 & 0\\ \hline
                    4 & 2 & 1 & 1\\ \hline
                    5 & 3 & 2 & 1\\ \hline
                    6 & 2 & 1 & 1\\ \hline
                    7 & 2 & 1 & 1\\ \hline
                    8 & 2 & 1 & 1\\ \hline
                \end{tabular} 
                \caption{Información de \textit{cortes} por equipos}
            \end{table}
            
            Por otro lado, las figuras 6-14 presentan el horario de cada equipo.\\
            Con la fecha y consecuentemente el equipo al cual se enfrentará.
            
            \begin{table}[H]
                \centering
                \begin{tabular}{|c||c|}
                    \hline
                    Fecha & Equipo\\ \hline
                    1 &  \\ \hline
                    2 & Atlético Nacional \\ \hline
                    3 & América de Cali \\ \hline
                    4 & Junior \\ \hline
                    5 & Millonarios \\ \hline
                    6 & Deportivo Cali \\ \hline
                    7 & Patriotas \\ \hline
                    8 &  \\ \hline
                    9 & Deportivo Independiente Medellín \\ \hline
                    10 & Boyacá Chicó \\ \hline
                    11 &  \\ \hline
                \end{tabular} 
                \caption{Horario de Santa Fe}
            \end{table}

            \begin{table}[H]
                \centering
                \begin{tabular}{|c||c|}
                    \hline
                    Fecha & Equipo\\ \hline
                    1 & Deportivo Independiente Medellín \\ \hline
                    2 & Boyacá Chicó \\ \hline
                    3 & Patriotas \\ \hline
                    4 & Deportivo Cali \\ \hline
                    5 & Santa Fe \\ \hline
                    6 & Atlético Nacional \\ \hline
                    7 &  \\ \hline
                    8 & América de Cali \\ \hline
                    9 & Junior \\ \hline
                    10 &  \\ \hline
                    11 &  \\ \hline
                \end{tabular} 
                \caption{Horario de Millonarios}
            \end{table}

            \begin{table}[H]
                \centering
                \begin{tabular}{|c||c|}
                    \hline
                    Fecha & Equipo\\ \hline
                    1 & América de Cali \\ \hline
                    2 & Santa Fe \\ \hline
                    3 & Deportivo Independiente Medellín \\ \hline
                    4 &  \\ \hline
                    5 & Deportivo Cali \\ \hline
                    6 & Millonarios \\ \hline
                    7 & Junior \\ \hline
                    8 & Patriotas \\ \hline
                    9 & Boyacá Chicó\\ \hline
                    10 &  \\ \hline
                    11 &  \\ \hline
                \end{tabular} 
                \caption{Horario de Atlético Nacional}
            \end{table}

            \begin{table}[H]
                \centering
                \begin{tabular}{|c||c|}
                    \hline
                    Fecha & Equipo\\ \hline
                    1 & Millonarios \\ \hline
                    2 & Deportivo Cali \\ \hline
                    3 & Atlético Nacional \\ \hline
                    4 & Boyacá Chicó \\ \hline
                    5 & América de Cali \\ \hline
                    6 & \textcolor{white}{Deportivo Independiente Medellín} \\ \hline
                    7 &  \\ \hline
                    8 &  \\ \hline
                    9 & Santa Fe\\ \hline
                    10 & Junior \\ \hline
                    11 & Patriotas \\ \hline
                \end{tabular} 
                \caption{Horario de Deportivo Independiente Medellín}
            \end{table}

            \begin{table}[H]
                \centering
                \begin{tabular}{|c||c|}
                    \hline
                    Fecha & Equipo\\ \hline
                    1 & Junior \\ \hline
                    2 & Deportivo Independiente Medellín \\ \hline
                    3 &  \\ \hline
                    4 & Millonarios \\ \hline
                    5 & Atlético Nacional \\ \hline
                    6 & Santa Fe \\ \hline
                    7 &  \\ \hline
                    8 & Boyacá Chicó \\ \hline
                    9 & Patriotas\\ \hline
                    10 & América de Cali \\ \hline
                    11 &  \\ \hline
                \end{tabular} 
                \caption{Horario de Deportivo Cali}
            \end{table}

            \begin{table}[H]
                \centering
                \begin{tabular}{|c||c|}
                    \hline
                    Fecha & Equipo\\ \hline
                    1 & Atlético Nacional\\ \hline
                    2 &  \\ \hline
                    3 & Santa Fe \\ \hline
                    4 &  \\ \hline
                    5 & Deportivo Independiente Medellín \\ \hline
                    6 & Patriotas \\ \hline
                    7 & Boyacá Chicó \\ \hline
                    8 & Millonarios\\ \hline
                    9 & \\ \hline
                    10 & Deportivo Cali \\ \hline
                    11 & Junior \\ \hline
                \end{tabular} 
                \caption{Horario de América de Cali}
            \end{table}

            \begin{table}[H]
                \centering
                \begin{tabular}{|c||c|}
                    \hline
                    Fecha & Equipo\\ \hline
                    1 & Patriotas\\ \hline
                    2 & Millonarios \\ \hline
                    3 & Junior \\ \hline
                    4 & Deportivo Independiente Medellín \\ \hline
                    5 &  \\ \hline
                    6 &  \\ \hline
                    7 & América de Cali \\ \hline
                    8 & Deportivo Cali\\ \hline
                    9 & Atlético Nacional \\ \hline
                    10 & Santa Fe \\ \hline
                    11 &  \\ \hline
                \end{tabular} 
                \caption{Horario de Boyacá Chicó}
            \end{table}

            \begin{table}[H]
                \centering
                \begin{tabular}{|c||c|}
                    \hline
                    Fecha & Equipo\\ \hline
                    1 & Boyacá Chicó\\ \hline
                    2 & Junior \\ \hline
                    3 & Millonarios \\ \hline
                    4 &  \\ \hline
                    5 &  \\ \hline
                    6 & América de Cali \\ \hline
                    7 & Santa Fe \\ \hline
                    8 & Atlético Nacional\\ \hline
                    9 & Deportivo Cali \\ \hline
                    10 &  \\ \hline
                    11 & Deportivo Independiente Medellín \\ \hline
                \end{tabular} 
                \caption{Horario de Patriotas}
            \end{table}

            \begin{table}[H]
                \centering
                \begin{tabular}{|c||c|}
                    \hline
                    Fecha & Equipo\\ \hline
                    1 & Deportivo Cali\\ \hline
                    2 & Patriotas \\ \hline
                    3 & Boyacá Chicó \\ \hline
                    4 & Santa Fe \\ \hline
                    5 &  \\ \hline
                    6 &  \\ \hline
                    7 & Atlético Nacional \\ \hline
                    8 & \\ \hline
                    9 & Millonarios \\ \hline
                    10 & Deportivo Independiente Medellín \\ \hline
                    11 & América de Cali \\ \hline
                \end{tabular} 
                \caption{Horario de Junior}
            \end{table}
                    
            \section{Conclusión}
            Con la implementación de ciertos conceptos de Teoría de Grafos, restricciones fuertes y restricciones suaves, se logró generar un horario para la Liga Profesional Colombiana de Fútbol para 9 equipos 
            que es factible y óptimo en el sentido de minimizar el número de \textit{cortes} en el horario.
            
            Adicionalmente, se pueden derivar 2 conclusiones claves:
            \begin{enumerate}
                \item En comparación con un torneo todos contra todos sencillo de \textit{n} equipos sin restricciones, el número de fechas para completarlo es \textit{n - 1} si \textit{n} es par y \textit{n} si \textit{n} es impar (donde cada equipo tiene una fecha de descanso).
                En nuestro caso, sin restricciones necesitariamos \textit{9} fechas, sin embargo, obtuvimos que se necesitan \textit{11} fechas. Es decir \textbf{al implementar ciertas restricciones al torneo, se incrementa el número de fechas en total y número de fechas de descanso por equipo}.
                Teniendo en cuenta el punto de vista de la Dimayor organización (entidad encargada de administrar y reglamentar los torneos de fútbol profesional en Colombia), 
                esto se puede ver como un aspecto negativo ya que se requiere hacer el torneo con el mínimo número de fechas posibles y sin tantas fechas de descanso.
                \item Adicionalmente, en un torneo sencillos sin restricciones el mínimo número de \textit{cortes} es \textit{n - 2} si \textit{n} es par y \textit{0} si \textit{n} es impar \cite{de1988some}.  
                En los resultados obtuvimos que los \textit{cortes} por equipo varian entre 0 y 3, lo que significa que \textbf{al implementar ciertas restricciones al torneo, se incrementa el número de \textit{cortes} por equipo}.
                Esto obviamente es un aspecto negativo del horario generado debido a que crea una desventaja entre los equipos de la liga.             
            \end{enumerate}
            
            Esto significa que la implementación de restricciones a un torneo todos contra todos sencillos cambia ciertos aspectos que son considerados importantes tanto a las directivas como a los equipos que forman la liga.
            Aunque la Teoría de Grafos es útil al solucionar este tipo de problemas, y efectivamente genera una solución para problemas relativamente pequeños, es mejor usar otras herramientas con más poder computacional de 
            tal forma que se pueda generar horarios para ligas con más equipos y poder abordar todas las restricciones necesarias.  
            
        \bibliographystyle{apacite}
        \bibliography{Referencias}
        \nocite{lewis2011application}
        \nocite{de1990constrained}
        \nocite{hoshino2013graph}
    \end{multicols}
\end{document}