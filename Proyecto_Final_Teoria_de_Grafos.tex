\documentclass[11pt]{article}

\renewcommand{\thesection}{\Roman{section}} 
\renewcommand{\thesubsection}{\thesection-\Alph{subsection}}
\usepackage{hyperref}
\usepackage{sectsty}
\usepackage{multicol}
\usepackage[margin=1in]{geometry}
\usepackage{amsfonts, amsmath, amssymb}
\usepackage[none]{hyphenat}
\usepackage{fancyhdr}
\usepackage{parskip}
\usepackage{float}
\usepackage[nottoc, notlot, notlof]{tocbibind}
\usepackage{graphicx}
\usepackage{apacite}
\usepackage{xcolor}

\pagestyle{fancy}
\fancyhead{}
\fancyfoot{}
\fancyhead[L]{\slshape{Proyecto Final Teoría de Grafos 2021-1}}
\fancyfoot[C]{\thepage}
%\renewcommand{\headrulewidth}{0pt}
%\renewcommand{\baselinestretch}{1.5}
\allsectionsfont{\centering}


\begin{document}
    \begin{center}
        \huge{\textbf{Optimización del torneo colombiano de fútbol mediante el coloreo de grafos}}\\[10pt]
        \small{\textbf{Santiago Alvarez Barbosa y Nicolás Velandia Sanabria}}\\[10pt]
        \small{\textbf{Universidad del Rosario, Matemáticas Aplicadas y Ciencias de la Computación}}\\[10pt]
        \today\\
        \rule{\textwidth}{0.5pt}
            \begin{abstract}
                \textcolor{red}{\textbf{\MakeUppercase{revisar ortografía y poner ciertas parte en modo matematico }}}\\
                Repositorio Github: \url{https://github.com/SantiagoAlvarezb/Graph_Theory_Final_Project}
            \end{abstract}
        \rule{\textwidth}{0.5pt}
    \end{center}

    \begin{multicols}{2}
    \setcounter{page}{1}

        \section{Introducción}
            La coloración de grafos considera la etiquetación de vértices o aristas de un grafo G con diferentes colores. 
            Se colorea de tal manera que ningún par de vértices o par de aristas adyacentes tengan el mismo color.\\
            Frecuentemente, la aplicación de coloración de grafos se ve en problemas involucrando horarios colegiales o 
            universitarios, donde su finalidad es poder crear un horario óptimo donde se cumplen todas las restricciones consideradas. 
            Teniendo esto en cuenta, se puede aplicar las técnicas de coloración de grafos en otros ámbitos similares. 
            Un ejemplo de esto es dentro del ámbito deportivo, específicamente en torneos todos contra todos. La mayoría de las 
            ligas deportivas del mundo se basan en estos torneos, que consisten en \textit{n} equipos, donde se requiere que cada equipo se 
            enfrente a los demás m veces dentro de un número específico de rondas. Por lo tanto, podemos aplicar ciertas 
            restricciones a ligas deportivas actuales, y con el uso de la coloración de grafos, obtener una solución óptima 
            del horario deportivo. 

        \section{Problema}
            Nos enfocaremos en la liga profesional colombiana (FPC), el cual es un torneo todos contra todos sencillo 
            (cada equipo juega contra el otro una vez), en la temporada 2021-1. Construiremos el grafo de tal forma que podamos 
            usar algoritmos para encontrar el número cromático, que en nuestro caso es el número mínimo de rondas que el torneo necesita.
            Adicional al problema de coloreo general, consideraremos ciertas restricciones reales que se pueden incorporar al grafo.\\[10pt]
            Ejemplos de restricciones que podríamos implementar:

            \begin{itemize}
                \item Minimizar el número de cortes en el horario, es decir, minimizar el número de veces que un equipo juega local o de visitante consecutivamente
                \item Evitar que ciertos partidos se jueguen en la misma fecha debido a que equipos pueden compartir estadio
                \item Equipo que empieza de local debe terminar la temporada como visitante y viceversa
            \item Ciertos equipos deben jugar en ciertas rondas. Un ejemplo de porque se aplica esta restricción es para incrementar los ingresos de televisión.
            \end{itemize}
   
            \subsection{Objetivos}
                Nuestro objetivo general es encontrar un horario válido para la liga profesional colombiana en la temporada 2021-A. Con el uso de la coloración 
                de grafos y sus técnicas.\\[10pt]
        
                Objetivos específicos:
                \begin{itemize}
                    \item Encontrar una forma de representar las restricciones especificadas 
                    \item Encontrar el número cromático de nuestro grafos con el uso de algoritmos 
                \end{itemize}

        \section{Marco Teórico}

            \begin{itemize}
                \item Grafo: Un grafo G es una terna que consiste en un conjunto de vértices V(G), un conjunto 
                de aristas E(G) y una relación que asocia a cada arista con un par de vértices (extremos) 
                no necesariamente distintos
                \item Grafo Simple: G = (V,E) es un grafo sin bucles ni aristas múltiples, donde E es un conjunto 
                de pares no ordenados de vértices
                \item Número cromático: El número cromático X(G) es el mínimo número de colores necesarios 
                para etiquetar los vértices de G de tal manera que vértices adyacentes reciben colores 
                distintos
                \item Subgrafo Inducido: Es un subgrafo que se obtiene al eliminar un conjunto de vértices. Se 
                escribe G[T] para G-T’ donde T’ = V(G) - T, este es el grafo inducido por T                 
            \end{itemize}
        
        \section{Modelamiento}
            Como nos enfocaremos en la liga colombiana de fútbol en la temporada 2021-1, hay que 
            tener en cuenta ciertos aspectos para la construcción de nuestro grafo
            \begin{itemize}
                \item  Hay 19 equipos
                \item  Hay 17 estadios (2 pares de equipos comparten estadio)
                \item  En el torneo actual hay 4 partidos denominados clásicos
                \begin{itemize}
                    \item  Millonarios vs. Santa Fe
                    \item  Atlético Nacional vs. Deportivo Independiente Medellín
                    \item America de Cali vs Deportivo Cali
                    \item  Boyacá Chicó - Patriotas
                \end{itemize}
            \end{itemize}

            Ya con esto en cuenta podemos implementar un grafo general G, el cual se usará para 
            encontrar el número cromático, es decir, encontrar el mínimo número de fechas para la 
            realización del torneo.\\
            Para este grafo, los vértices serán etiquetados como un par ordenado {a,b} que indican un 
            partido entre el equipo a y el equipo b, con a siendo local; Mientras que las aristas unen 
            vértices que tienen equipos en común.Al finalizar el procesamiento del grafo G, nos enfocaremos en los grafos inducidos por el 
            coloreo para así implementar las restricciones.\\[10pt]
            Tendremos en cuenta restricciones fuertes (las cuales se debe cumplir) y restricciones 
            suaves (las cuales se cumplirán en lo posible). Estas restricciones son restricciones que 
            actualmente se usan en el FPC, adicionalmente, implementaremos algunas que equipos y 
            dirigentes de clubes piden a la Dimayor para mejorar la liga. Ya implementadas las 
            restricciones, volvemos a usar coloración de grafos para ver, dentro de las jornadas, como 
            se puede organizar los juegos.\\[10pt]
            Condiciones fuertes:
            \begin{itemize}
                \item Equipos que comparten estadios no pueden jugar en la misma jornada como locales
                \item NO hay fechas de clásicos. Usualmente había una fecha predeterminada donde 
                todos los clásicos se juegan. En este caso no se tendrá en cuenta
            \end{itemize}
            Condiciones suaves:
            \begin{itemize}  
                \item Minimizar fechas consecutivas donde un equipo juega como visitante o local 2 veces 
                seguidas. Por lo tanto, tener en lo posible un torneo donde para cada equipo se tiene 
                el formato Local-Visitante-Local-Visitante… etc.     
            \end{itemize}

        \section{Soluciones Propuestas}
            Proponemos usar un algoritmo de coloreo para hallar el numero cromático (n), para 
            posteriormente construir n grafos con sus respectivas restricciones, y de esta forma 
            hallar el horario óptimo para la realización adecuada del torneo.\\
            Adicionalmente agregaremos restricciones al grafo obtenido, y posteriormente aplicar un 
            algoritmo de coloreo para hallar los equipos que juegan cada jornada y así llegar al 
            horario óptimo para la realización del torneo.
            
        \bibliographystyle{apacite}
        \bibliography{Referencias}

    \end{multicols}
\end{document}